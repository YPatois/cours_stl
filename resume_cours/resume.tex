\documentclass{article}
\usepackage[utf8]{inputenc}
\usepackage[french]{babel}
\usepackage{amsmath}
\usepackage{siunitx}

\DeclareSIUnit{\nothing}{\relax}

\title{Résumé du cours du 15 octobre 2025}
\author{Yannick Patois}
\date{17/10/2025}

\begin{document}

\maketitle

\section{Rappel sur les puissances de nombres relatifs}

Les puissances entières sont un concept fondamental en mathématiques, utilisées pour représenter une multiplication
répétée d'un nombre par lui-même. Nous nous limitons ici aux exposants entiers relatifs.

\subsection{Puissances positives}

La puissance positive est la forme la plus courante et la plus intuitive des puissances entières. 

\[
a^n = \underbrace{a \times a \times \dots \times a}_{n}  \text{ fois}
\]

Où \( n \) est un entier positif.
\vfill
\textbf{Exemple :}
\[
2^3 = 2 \times 2 \times 2 = 8
\]

Comme vous le voyez, la puissance positive consiste simplement à multiplier le nombre \( a \) par lui-même \( n \) fois.

\subsection{Puissances négatives}

Les puissances négatives sont définies comme l'inverse de la puissance positive correspondante.

\[
a^{-n} = \frac{1}{a^n}
\]
\vfill
\textbf{Exemple :}
\[
2^{-2} = \frac{1}{2^2} = \frac{1}{4}
\]

Ici, \( 2^{-2} \) est l'inverse de \( 2^2 \), ce qui donne \( \frac{1}{4} \).

\subsection{Puissances à l'exposant zéro}

Un cas particulier est celui où l'exposant est zéro. Pour tout nombre non nul, on a :

\[
a^0 = 1
\]

\textbf{Explication :}
Cette règle peut sembler arbitraire au premier abord, mais elle est nécessaire pour maintenir
la cohérence des opérations mathématiques.

\section{Puissances de 10 entières}

Les puissances de 10 sont particulièrement importantes en mathématiques et en sciences,
notamment pour exprimer des nombres très grands ou très petits.

\subsection{Puissances positives de 10}

 Une puissance positive de 10 s'écrit sous la forme \( 10^n \), où \( n \) est un entier positif. Chaque augmentation de l'exposant correspond à l'ajout d'un zéro supplémentaire après le 1.

 \[
\begin{aligned}
10^0 &= &0 \\
10^1 &= &10 \\
10^2 &= &100 \\
10^3 &= &1000 \\
&\vdots
\end{aligned}
\]
\vfill
\textbf{Exemples :}
\begin{align*}
10^4 &= 10000 \quad (\text{4 zéros}) \\
10^5 &= 100000 \quad (\text{5 zéros}) \\
\end{align*}

\subsection{Puissances négatives de 10}

 Une puissance négative de 10 s'écrit sous la forme \( 10^{-n} \), où \( n \) est un entier positif. Elle représente une valeur fractionnaire avec \( n-1 \) zéros après la virgule décimale, suivie d'un 1.

\begin{align*}
10^{0} &= 0 \\
10^{-1} &= 0,1 \\
10^{-2} &= 0,01 \\
10^{-3} &= 0,001 \\
&\vdots
\end{align*}
\vfill
\textbf{Exemples :}
\begin{align*}
10^{-4} &= 0,0001 \quad (\text{3 zéros après la virgule}) \\
10^{-5} &= 0,00001 \quad (\text{4 zéros après la virgule}) \\
\end{align*}

\section{Préfixes multiplieurs SI}

Les préfixes multiplieurs SI sont utilisés pour exprimer des puissances de 10 de manière concise.
Ils permettent de représenter des nombres très grands ou très petits en associant 
un symbole à une puissance de 10 spécifique.

\begin{center}
\begin{tabular}{lll}
\hline
Symbole            & Nom              & Valeur multiplicative\\
\hline
\si{\kilo\nothing} & kilo             & \(10^3\) \\
\si{\mega\nothing} & mega             & \(10^6\) \\
\si{\giga\nothing} & giga             & \(10^9\) \\
\si{\tera\nothing} & tera             & \(10^{12}\) \\
\hline
\si{\milli\nothing} & milli            & \(10^{-3}\) \\
\si{\micro\nothing} & micro            & \(10^{-6}\) \\
\si{\nano\nothing}  & nano             & \(10^{-9}\) \\
\si{\pico\nothing}  & pico             & \(10^{-12}\) \\
\hline
\end{tabular}
 \end{center}

\textbf{Exemples :}
\begin{itemize}
    \item \SI{1}{\kilo\metre} = \(10^3\) mètres = 1000 mètres
    \item \SI{1}{\micro\second} = \(10^{-6}\) secondes
    \item \SI{1}{\giga\hertz} = \(10^9\) hertzs
\end{itemize}

Ce système de préfixes est particulièrement utile pour exprimer des grandeurs physiques de manière concise et lisible.

\section{Le Nombre d'Avogadro}

Le nombre d'Avogadro, nommé en l'honneur du scientifique italien Amedeo Avogadro,
est une constante physique fondamentale dans les domaines de la chimie et de la physique. Il est utilisé pour 
relier les quantités macroscopiques aux quantités microscopiques.

\subsection{Définition}

Le nombre d'Avogadro est historiquement défini comme le nombre d'entités (atomes, molécules, ions, etc.) 
ontenus dans une mole d'une substance. Sa valeur était approximativement de :
\[
N_A = 6,022 \times 10^{23} \text{ entités/mol}
\]

Depuis 2020, le nombre d'Avogadro est une constante définie exactement comme:

\[
N_A = 6,022 140 76 \times 10^{23} \text{ entités/mol}
\]

Cette valeur est extrêmement importante en chimie, notamment dans les calculs de concentration,
de quantité de matière, etc.

\section{Notation des symboles chimiques}

Dans le tableau périodique, chaque élément chimique est représenté par un symbole, généralement
une ou deux lettres. Ce symbole est souvent accompagné de deux nombres : le numéro atomique 
(\(Z\)) et le numéro de masse (\(A\)). Ces informations sont représentées sous la forme :

\[
^{A}_{Z}X
\]

où :
- \(Z\) est le numéro atomique, qui correspond au nombre de protons dans le noyau de l'atome.
- \(A\) est le numéro de masse, qui correspond au nombre total de protons et de neutrons (les nucléons) dans le noyau.
- \(X\) est le symbole de l'élément.
\vfill
\textbf{Exemple}
Pour l'élément carbone :
\[
^{12}_{6}C
\]
- \(Z = 6\) : il y a 6 protons dans le noyau.
- \(A = 12\) : il y a 6 neutrons (12 - 6 = 6).
- \(X = C\) : symbole du carbone.

\section{Notation des isotopes}

\textbf{Definition} : Les isotopes sont des atomes d'un élément chimique ayant des numéros de masse (\(A\)) différents
mais des numéros atomiques (\(Z\)) identiques, en raison d'un nombre variable de neutrons.

\vfill
\textbf{Exemples}
Pour l'élément hydrogène :
- Le "protium" (isotope le plus commun, généralement juste appelé "hydrogène" ou "proton") : \({}^{1}_{1}H\)
- Le deutérium : \({}^{2}_{1}H\)
- Le tritium : \({}^{3}_{1}H\)

Ces isotopes ont tous le même symbole (\(H\)) et le même numéro atomique (\(Z = 1\)),
mais des numéros de masse différents (\(A = 1, 2, 3\)).

\end{document}
