\documentclass{article}
\usepackage[version=4]{mhchem}
\usepackage{chemfig}
\usepackage{mol2chemfig}

\title{Correction des exercices}
\author{Yannick Patois}
\date{12/11/2025}

\begin{document}

\section{Réponse aux Exercices}

\subsection**{Exercice 1 : L'acide citrique}

L'acide citrique a pour formule brute \ce{C6H8O7}. Voici sa composition en atomes~:

\begin{itemize}
    \item Carbone (C) : Il y a 6 atomes de carbone.
    \item Hydrogène (H) : Il y a 8 atomes d'hydrogène.
    \item Oxygène (O) : Il y a 7 atomes d'oxygène.
\end{itemize}

\subsection*{Exercice 2 : La dexaméthasone}

La dexaméthasone est composée de~:
\begin{itemize}
    \item 22 atomes de carbone,
    \item 29 atomes d'hydrogène,
    \item 1 atome de fluor,
    \item 5 atomes d'oxygène.
\end{itemize}

Sa formule brute est donc : \ce{C22H29FO5}.

\subsection*{Exercice 3 : La L-glutamine}

\textbf{Formule semi-développée :}
\[
\ce{NH2-CO-CH2-CH2-CH(NH2)-COOH}
\]

On peut développer davantage (comme fait dans le 4):
\[
    \chemfig{
    NH_2-C([2]=O)-CH_2-CH_2-CH(-[2]NH_2)-C([2]=O)(-[6]OH)
    }
\]


\subsection*{Exercice 4 : Le géraniol}

\[
    \chemfig{
        C(-[2]H)(-[4]H)(-[6]H)-[,1.4]C(-[2]C(-[0]H)(-[2]H)(-[4]H))%
        =C(-[6]H)-C(-[2]H)(-[6]H)-C(-[2]H)(-[6]H)-[,1.4]C(-[2]C(-[0]H)(-[2]H)(-[4]H))=C%
        (-[6]H)-C(-[2]H)(-[6]H)-O-H
    }
\]

\subsection*{Exercice 5 : Un colorant rouge}

On compte juste les atomes sur le rouge para~:

\begin{itemize}
    \item Carbone (C) : Il y a 16 atomes de carbone.
    \item Hydrogène (H) : Il y a 11 atomes d'hydrogène.
    \item Oxygène (O) : Il y a 3 atomes d'oxygène.
    \item Azote (N) : Il y a 3 atomes d'azote.
\end{itemize}

Et donc: \ce{C16H11O3N3}


Note: la molécule posède des sites chargés, mais elle reste globalement neutre.

Formule brute de l'ion diazonium:
\begin{itemize}
    \item Carbone (C) : Il y a 6 atomes de carbone.
    \item Hydrogène (H) : Il y a 4 atomes d'hydrogène.
    \item Oxygène (O) : Il y a 2 atomes d'oxygène.
    \item Azote (N) : Il y a 3 atomes d'azote.
\end{itemize}

Et donc: \ce{C6H4O2N3+}

Ici, on a bien un ion, avec une charge plus excédentaire (un électron en moins).


\subsection*{Exercice 6 : Des molécules aux propriétés désinfectantes}

\chemfig{OH-[:210,,1]-[:150]}\\
Éthanol.

\smallskip

\chemfig{OH-[:210,,1](-[:150])-[:270]}\\
Isopropanol

\smallskip

\chemfig{O=[:330]-[:30]-[:330]-[:30]-[:330]=[:30]O}\\
Glutaraldéhyde

\subsection*{Exercice 7 : L’odeur des vieux livres}

% mol2chemfig -w -z -y delete -i pubchem 7720

\chemfig{HO-[:30,,2]-[:330](-[:270]-[:210])-[:30]-[:330]-[:30]-[:330]}\\
2-éthylhexanol


\subsection*{Exercice 8 : Principes actifs}


\chemfig{OH-[:90,,1]C=^[:30,,,1]CH-[:90,,1,1]CH=^[:150,,1]C(-[:210,,,2]HC%
=^[:270,,2,2]HC-[:330,,2]\phantom{C})-[:90,,,2]HN-[:30,,2]C(=[:330]O)%
-[:90,,,1]CH_3}\\
Paracétamol.

\smallskip

% mol2chemfig -w -z -c -y delete -i pubchem 3821

\chemfig{Cl-[:100]C-[:160]C(=_[:100,,,2]HC-[:40,,2]\mcfabove{C}{H}%
=_[:340,,,1]CH-[:280,,1,1]CH=_[:220,,1]\phantom{C})-[:220]C(%
-[:120]\mcfabove{N}{H}-[:180,,,2]H_3C)-[:330,,,1]CH_2-[:270,,1,1]CH_2%
-[:210,,1]\mcfbelow{C}{\mcfright{H}{_2}}-[:150,,,2]H_2C-[:90,,2]C(%
-[:30]\phantom{C})=[:150]O}\\
Kétamine

\smallskip

Formule brute de l'ibuprofene~:
\[
    \ce{C13H18O2}
\]

\subsection*{Exercice 9 : Autour d’une bouteille de rhum !}

\chemfig{O=[:90](-[:150])-[:30]O-[:330]-[:30]}\\
Acétate d'éthyle

Donc (a) et (c) (même molécule, dans l'autre sens).

\smallskip

% ol2chemfig -w -z -c -y delete -i pubchem 10953545
\chemfig{HO-[:30,,2]C(=[:90]O)-[:330]\mcfabove{C}{H}(-[:30]\mcfbelow{C}{H}(%
-[:90,,,1]CH_3)-[:330,,,1]CH_3)-[:270,,,1]CH_2-[:210,,1,2]H_3C}\\
Acide 2-éthyl-3--méthylbutanoïque.

\smallskip

Un acide "truc"-oïque porte un groupe carboxyle (\ce{-COOH}) en extrémité de la chaîne.
Comme c'est la fonction principale, la numérotation commence sur ce carbone.
La chaîne principale fait 4 atomes (butan-)
Sur le carbone 2, il porte un groupe éthyle (\ce{-CH2-CH3}) et un groupe 
méthyle (\ce{-CH3}) sur le carbone 3.
C'est donc le (b).

\subsection*{Exercice 10 : Nomenclature des alcanes}

\begin{itemize}
    \item a: butane \chemfig{-[:330]-[:30]-[:330]}
    \item b: 4-éthylheptane \chemfig{-[:30]-[:90](-[:30]-[:330]-[:30])-[:150]-[:210]-[:150]}
    \item c: 2,3-diméthylpentane \chemfig{-[:90](-[:30](-[:90])-[:330])-[:150]-[:210]}
    \item d: 3,4-diméthylhexane \chemfig{-[:90](-[:150]-[:210])-[:30](-[:90])-[:330]-[:30]}
\end{itemize}

Note: le 3,4-diméthylhexane possède 3 stéréoisomères (RR, SS et RS/SR).

\subsection*{Exercice 11 : Des solvants organiques}

\begin{itemize}
    \item a: groupe carboxyle (acide -oïque)
    \item b: groupe hydroxyle (-anol)
    \item c: groupe carbonyle, cétone (-one)
    \item d: groupe carbonyle, cétone (-one)
    \item e: groupe amine (-amine)

\end{itemize}

\subsection*{Exercice 12 : Nomenclature}
\smallskip
\chemfig{H_2N-[:330,,2](-[:270])-[:30]-[:330]}\\
Butan-2-amine

\smallskip

\chemfig{OH-[:150,,1]-[:210](-[:270])-[:150]-[:210]}\\
2-Méthylbutan-1-ol

\smallskip

\chemfig{-[:90](-[:150]-[:210])-[:30]-[:330]-[:30]-[:330]}\\
3-Méthylheptane

\smallskip

\chemfig{O=[:210]-[:150](-[:90])-[:210](-[:270])-[:150]-[:210]}\\
2,3-diméthylpentanal

\smallskip

\chemfig{O=[:270](-[:210])-[:330](-[:270]-[:210])-[:30](-[:90])-[:330]%
-[:30]}\\
3-éthyl-4-méthylhexan-2-one

\smallskip

\chemfig{OH-[:150,,1](=[:90]O)-[:210](-[:150])-[:270]}\\
Acide 2-méthylpropanoïque (acide isobutyrique)

\end{document}
