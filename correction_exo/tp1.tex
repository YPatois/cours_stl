\documentclass{article}
\usepackage{chemfig}
\usepackage[version=4]{mhchem}

\title{Correction du TP1}
\author{Yannick Patois}
\date{05/11/2025}

\begin{document}

\section*{II/ Géométrie autour d’un atome tétragonal}

Un atome est tétragonal s’il possède 4 atomes voisins avec lesquels il forme des liaisons covalentes.  
Étude de la molécule de méthane de formule brute \ce{CH4}.

\subsection*{1. Construire le modèle moléculaire éclaté}

Le modèle moléculaire éclaté du méthane consiste en un atome central de carbone \ce{C} entouré de 
quatre atomes d'hydrogène \ce{H}. Dans ce modèle, chaque liaison covalente est représentée par une 
connexion entre les nuages électroniques de deux atomes.

\subsection*{2. Quelle est la géométrie autour de l’atome de carbone ?}

La géométrie autour de l'atome de carbone dans le méthane est \textbf{tétraédrique}. 
Les quatre liaisons \ce{C-H} s'organisent dans l'espace pour former une structure tétraédrique, où chaque angle 
entre les liaisons est d’environ 109,5°.

\subsection*{3. Déterminer son schéma de Lewis}

Le schéma de Lewis du méthane est représenté comme suit~:

\begin{itemize}
\item Un atome de carbone central.
\item Quatre atomes d'hydrogène autour du carbone.
\item Chaque liaison \ce{C-H} est une liaison covalente simple, avec une paire d'électrons partagée entre 
le carbone et chaque hydrogène.
\end{itemize}

\chemfig{C(-[0]H)(-[2]H)(-[4]H)(-[6]H)}

\subsection*{4. Comment peut-on expliquer cette géométrie ?}

La géométrie tétraédrique du méthane s'explique par la répulsion entre les paires d'électrons dans 
la théorie VSEPR (Valence Shell Electron Pair Repulsion). Le carbone possède quatre paires 
d'électrons de liaison, toutes identiques et sans paire d'électrons non liantes. 
Ces paires d'électrons se repoussent mutuellement et s'organisent dans l'espace pour minimiser cette répulsion, 
ce qui conduit à une disposition tétraédrique.

\subsection*{5. Déterminer la représentation de Cram de cette molécule}

La représentation de Cram du méthane montre le carbone central avec ses quatre liaisons simples vers 
les atomes d'hydrogène. Étant donné que toutes les liaisons sont identiques et qu'il n'y a 
pas de stéréoisomérie, la structure est représentée de manière symétrique.

\chemfig{
    C(-[:90]H)(-[:-150]H)(<[:-60]H)(<:[:-30]H)
}

\subsection*{Conclusion}

Le méthane \ce{CH4} est une molécule tétraédrique où le carbone central est lié à quatre atomes 
d'hydrogène par des liaisons covalentes simples. Cette géométrie est expliquée par la théorie VSEPR, 
et les représentations de Lewis et de Cram mettent en évidence cette structure symétrique.

\section*{III/ Géométrie autour d’un atome trigonal}

Un atome est trigonal s’il possède 3 atomes voisins avec lesquels il forme des liaisons covalentes.
Étude de la molécule d'ammoniac de formule brute \ce{NH3}.

\subsection*{1. Construire le modèle moléculaire éclaté}

Le modèle moléculaire éclaté de l'ammoniac se compose d'un atome central d'azote entouré de trois atomes
 d'hydrogène. Chaque liaison covalente est représentée par une connexion entre les nuages 
électroniques de l'azote et des hydrogènes.

\subsection*{2. Quelle est la géométrie autour de l’atome d’azote ?}

La géométrie autour de l'atome d'azote dans l'ammoniac est \textbf{trigonal pyramidale}. 
Les trois liaisons \ce{N-H} et la paire d'électrons libre s'organisent selon une géométrie tétraédrique, 
ce qui conduit à une forme moléculaire trigonal pyramidale.

\subsection*{3. Déterminer son schéma de Lewis}

Le schéma de Lewis de l'ammoniac est représenté comme suit~:

\begin{itemize}
    \item Un atome central d'azote.
    \item Trois atomes d'hydrogène liés à l'azote par des liaisons covalentes simples.
    \item Une paire d'électrons libre sur l'azote.
\end{itemize}

La représentation en schéma de Lewis est~:

\chemfig{
    \charge{90=\|}{N}(-[0]H)(-[4]H)(-[6]H)
}

\subsection*{4. Comment expliquer cette géométrie ?}

Selon la théorie VSEPR (Valence Shell Electron Pair Repulsion), les paires électroniques (liantes et non liantes) s'éloignent pour minimiser les répulsions. Dans le cas de l'ammoniac, il y a 
quatre régions électroniques autour de l'azote~: trois liaisons \ce{N-H} et une paire d'électrons libre. Ces régions adoptent une géométrie tétraédrique. Cependant, comme seule la paire 
d'électrons libre ne contribue pas à la position des atomes, la forme moléculaire résultante est trigonal pyramidale.

\subsection*{5. Déterminer la représentation de Cram de cette molécule}

La représentation de Cram met en évidence la structure tridimensionnelle de l'ammoniac. Elle montre l'atome d'azote central avec les trois atomes d'hydrogène disposés dans un plan et la paire 
d'électrons libre dirigée vers le haut.

\chemfig{
    N(-[:-150]H)(<[:-60]H)(<:[:-30]H)
}

\subsection*{Conclusion}

L'ammoniac \ce{NH3} est une molécule à géométrie trigonal pyramidale. 
L'azote central est lié à trois atomes d'hydrogène par des liaisons covalentes simples, et la présence d'une paire 
d'électrons libre influe sur la forme tridimensionnelle de la molécule, 
expliquée par la théorie VSEPR. Les schémas de Lewis et de Cram illustrent 
clairement cette structure.


\section*{III/ Géométrie autour d’un atome trigonal}

Un atome est dit trigonal s’il possède trois substituants.
\subsection*{3.2/ L’atome ne porte pas de doublet libre}

Exemple~: Étude de la molécule de méthanal de formule brute \ce{CH2O}.

\subsection*{1. Construire le modèle moléculaire éclaté}

Le modèle moléculaire éclaté du méthanal représente le carbone central partageant des paires d'électrons 
avec deux hydrogènes et un oxygène. Le carbone est lié par une liaison double à 
l'oxygène, ce qui implique deux paires d'électrons partagées.

\subsection*{2. Quelle est la géométrie autour de l’atome de carbone ?}

La géométrie autour de l'atome de carbone dans le méthanal est \textbf{trigonal plane}. 
La forme moléculaire est trigonal plan. 
Cela signifie que les trois substituants (deux hydrogènes et un oxygène) sont disposés dans un plan avec 
des angles entre eux d'environ 120°.

\subsection*{3. Déterminer son schéma de Lewis}

Le schéma de Lewis du méthanal est représenté comme suit~:

\begin{itemize}
    \item Un atome de carbone central.
    \item Deux atomes d'hydrogène liés par des liaisons simples au carbone.
    \item Un atome d'oxygène doublement lié au carbone, avec deux paires d'électrons non partagées.
\end{itemize}

La représentation en schéma de Lewis est~:
\smallskip

\chemfig{
    C(=[2]\charge{45=\|,135=\|}{O})(-[4]H)(-[0]H)
}

\subsection*{4. Comment expliquer cette géométrie ?}

Selon la théorie VSEPR (Valence Shell Electron Pair Repulsion), les paires d'électrons cherchent 
à s'éloigner autant que possible pour minimiser les répulsions. Dans le méthanal, le carbone comporte une double-liaison,
ce qui conduit à garder trois groupes d'électrons autour du carbone. La forme moléculaire est donc 
une géométrie trigonal plan avec des angles de 120° entre les liaisons.

\subsection*{5. Déterminer la représentation de Cram de cette molécule}

La représentation de Cram du méthanal montre le carbone central avec ses trois substituants~: 
deux hydrogènes et un oxygène doublement lié, disposés dans un plan avec des angles de 120°.

\chemfig{
    C(=[2]O)(-[5]H)(-[7]H)
}

\subsection*{Conclusion}

Le méthanal \ce{CH2O} est une molécule où le carbone central adopte une géométrie trigonal plan due à 
la double liaison. Les deux hydrogènes et l'oxygène doublement lié sont disposés dans un plan 
avec des angles de 120° entre eux, conformément à la théorie VSEPR. Les schémas de Lewis 
et de Cram illustrent clairement cette structure planaire et symétrique.


\section*{IV/ Géométrie autour d’un atome digonal}

Un atome est dit digonal s’il possède deux atomes voisins avec lesquels il forme des liaisons covalentes.

\subsection*{4.1 L’atome porte deux doublets libres}

\textbf{Exemple~:} Étude de la molécule d’eau de formule brute \ce{H2O}.

\subsubsection*{1. Construire le modèle moléculaire éclaté}

Le modèle moléculaire éclaté de l'eau représente l'oxygène central partageant des paires d'électrons avec deux atomes d'hydrogène. L'oxygène possède également deux doublets libres.

\subsubsection*{2. Quelle est la géométrie autour de l’atome d’oxygène ?}

La géométrie autour de l'atome d'oxygène dans l'eau est \textbf{angulaire} ou en forme de V.

\subsubsection*{3. Déterminer son schéma de Lewis}

Le schéma de Lewis de l'eau est représenté comme suit~:

\begin{itemize}
    \item Un atome d'oxygène central.
    \item Deux atomes d'hydrogène liés par des liaisons simples à l'oxygène.
    \item Deux paires d'électrons non partagées sur l'oxygène.
\end{itemize}

La représentation en schéma de Lewis est~:

\medskip
\chemfig{
    H-[:52.24]\charge{45=\|,135=\|}{O}-[::-104.48]H
}

\subsubsection*{4. Comment expliquer cette géométrie ?}

Selon la théorie VSEPR (Valence Shell Electron Pair Repulsion), les paires d'électrons cherchent à s'éloigner autant que possible pour minimiser les répulsions. Dans l'eau, l'oxygène possède 
deux liaisons simples et deux doublets libres, ce qui conduit à une géométrie angulaire avec un angle entre les liaisons d'environ 104,5°.

\subsubsection*{5. Déterminer la représentation de Cram de cette molécule}

La représentation de Cram de l'eau montre l'oxygène central avec ses deux atomes d'hydrogène disposés en forme de V.

\chemfig{
    H-[:52.24]{O}-[::-104.48]H
}

\subsection*{4.2 L’atome ne porte pas de doublet libre}

\textbf{Exemple~:} Étude de la molécule de dioxyde de carbone de formule brute \ce{CO2}.

\subsubsection*{1. Construire le modèle moléculaire éclaté}

Le modèle moléculaire éclaté du dioxyde de carbone représente un atome de carbone central lié à deux atomes d'oxygène par des doubles liaisons covalentes. Le carbone ne possède pas de doublets 
libres.

\subsubsection*{2. Quelle est la géométrie autour de l’atome de carbone ?}

La géométrie autour de l'atome de carbone dans le dioxyde de carbone est \textbf{linéaire}.

\subsubsection*{3. Déterminer son schéma de Lewis}

Le schéma de Lewis du dioxyde de carbone est représenté comme suit~:

\begin{itemize}
    \item Un atome de carbone central.
    \item Deux atomes d'oxygène liés par des doubles liaisons au carbone.
    \item Aucune paire d'électrons non partagée sur le carbone.
\end{itemize}

La représentation en schéma de Lewis est~:

\chemfig{
    C(=\charge{45=\|,-45=\|}{O})(=[4]\charge{-135=\|,135=\|}{O})
}

\subsubsection*{4. Comment expliquer cette géométrie ?}

Selon la théorie VSEPR, les paires d'électrons cherchent à s'éloigner autant que possible pour minimiser les répulsions.
Dans le dioxyde de carbone, le carbone est lié à deux atomes d'oxygène 
par des doubles liaisons, ce qui conduit à une géométrie linéaire.

\section{Introduction}
La méthyléthylamine, de formule brute \ce{CH3NHCH2CH3}, est une amine secondaire.
 Elle possède un atome d'azote central lié à deux groupes alkyles (méthyle et éthyle) et à 
 un hydrogène, d'où son nom. Note: vous ne devez connaître la nomenclature que des amines primaires
(fonction amine en extrémité de la chaîne carbonée). Nous allons déterminer sa géométrie 
  moléculaire en nous basant sur le schéma de 
 Lewis et la théorie de la répulsion des paires d'électrons de valence (VSEPR), adaptée au niveau Terminale.

\section{Schéma de Lewis}
Le schéma de Lewis permet de représenter les liaisons et les doublests non liants autour des atomes.

L'atome d'azote (N) est central et possède 5 électrons de valence. Il forme :
\begin{itemize}
    \item Une liaison covalente avec le groupe méthyle \ce{-CH3},
    \item Une liaison covalente avec le groupe éthyle \ce{-CH2CH3},
    \item Une liaison covalente avec un hydrogène (N-H),
    \item Un doublet non liant (paire libre).
\end{itemize}

Le schéma de Lewis de la méthyléthylamine est le suivant :

\chemfig{CH_3-\charge{90=\|}{N}(-[6]H)-CH_2-CH_3}

\section{Théorie VSEPR}
La géométrie est très semblable à celle de l'ammoniac \ce{NH3} vu plus haut, pour les mêmes
raisons.

\chemfig{
    N(-[:-150]CH_3)(<[:-60]H)(<:[:-30]CH_2-CH_2)
}

On peut mettre deux liaisons dans le plan, pour la lisibilité:

\chemfig{
    N(-[:-150]CH_3)(<[:-90]H)(-[:-30]CH_2-CH_2)
}


En développant les carbones:

\smallskip
\chemfig{
    N(-[:-150]C(<[::80]H)(<:[::30]H)(-[::-60]H))%
    (<[:-90]H)(-[:-30]C(-[::60]C(<[::80]H)(<:[::30]H)(-[::-60]H))(<:[::-30]H)(<[::-60]H))
}
\smallskip

Représentation topologique:

\smallskip
\chemfig{
     [:-30]-[:30]-[:-30]N(-[:-90,.3]H)(-[:30])
}

\end{document}
