\documentclass{article}
\usepackage{chemfig}
\usepackage[version=4]{mhchem}

\title{Correction du TP1}
\author{Yannick Patois}
\date{05/11/2025}

\begin{document}

\section*{II/ Géométrie autour d’un atome tétragonal}

Un atome est tétragonal s’il possède 4 atomes voisins avec lesquels il forme des liaisons covalentes.  
Étude de la molécule de méthane de formule brute \ce{CH4}.

\subsection*{1. Construire le modèle moléculaire éclaté}

Le modèle moléculaire éclaté du méthane consiste en un atome central de carbone \ce{C} entouré de 
quatre atomes d'hydrogène \ce{H}. Dans ce modèle, chaque liaison covalente est représentée par une 
connexion entre les nuages électroniques de deux atomes.

\subsection*{2. Quelle est la géométrie autour de l’atome de carbone ?}

La géométrie autour de l'atome de carbone dans le méthane est \textbf{tétraédrique}. 
Les quatre liaisons \ce{C-H} s'organisent dans l'espace pour former une structure tétraédrique, où chaque angle 
entre les liaisons est d’environ 109,5°.

\subsection*{3. Déterminer son schéma de Lewis}

Le schéma de Lewis du méthane est représenté comme suit~:

\begin{itemize}
\item Un atome de carbone central.
\item Quatre atomes d'hydrogène autour du carbone.
\item Chaque liaison \ce{C-H} est une liaison covalente simple, avec une paire d'électrons partagée entre 
le carbone et chaque hydrogène.
\end{itemize}

\chemfig{C(-[0]H)(-[2]H)(-[4]H)(-[6]H)}

\subsection*{4. Comment peut-on expliquer cette géométrie ?}

La géométrie tétraédrique du méthane s'explique par la répulsion entre les paires d'électrons dans 
la théorie VSEPR (Valence Shell Electron Pair Repulsion). Le carbone possède quatre paires 
d'électrons de liaison, toutes identiques et sans paire d'électrons non liantes. 
Ces paires d'électrons se repoussent mutuellement et s'organisent dans l'espace pour minimiser cette répulsion, 
ce qui conduit à une disposition tétraédrique.

\subsection*{5. Déterminer la représentation de Cram de cette molécule}

La représentation de Cram du méthane montre le carbone central avec ses quatre liaisons simples vers 
les atomes d'hydrogène. Étant donné que toutes les liaisons sont identiques et qu'il n'y a 
pas de stéréoisomérie, la structure est représentée de manière symétrique.

\chemfig{
    C(-[:90]H)(-[:-150]H)(<[:-60]H)(<:[:-30]H)
}

\subsection*{Conclusion}

Le méthane \ce{CH4} est une molécule tétraédrique où le carbone central est lié à quatre atomes 
d'hydrogène par des liaisons covalentes simples. Cette géométrie est expliquée par la théorie VSEPR, 
et les représentations de Lewis et de Cram mettent en évidence cette structure symétrique.

\section*{III/ Géométrie autour d’un atome trigonal}

Un atome est trigonal s’il possède 3 atomes voisins avec lesquels il forme des liaisons covalentes.
Étude de la molécule d'ammoniac de formule brute \ce{NH3}.

\subsection*{1. Construire le modèle moléculaire éclaté}

Le modèle moléculaire éclaté de l'ammoniac se compose d'un atome central d'azote entouré de trois atomes
 d'hydrogène. Chaque liaison covalente est représentée par une connexion entre les nuages 
électroniques de l'azote et des hydrogènes.

\subsection*{2. Quelle est la géométrie autour de l’atome d’azote ?}

La géométrie autour de l'atome d'azote dans l'ammoniac est \textbf{trigonal pyramidale}. 
Les trois liaisons \ce{N-H} et la paire d'électrons libre s'organisent selon une géométrie tétraédrique, 
ce qui conduit à une forme moléculaire trigonal pyramidale.

\subsection*{3. Déterminer son schéma de Lewis}

Le schéma de Lewis de l'ammoniac est représenté comme suit~:

\begin{itemize}
    \item Un atome central d'azote.
    \item Trois atomes d'hydrogène liés à l'azote par des liaisons covalentes simples.
    \item Une paire d'électrons libre sur l'azote.
\end{itemize}

La représentation en schéma de Lewis est~:

\chemfig{
    \charge{90=\|}{N}(-[0]H)(-[4]H)(-[6]H)
}

\subsection*{4. Comment expliquer cette géométrie ?}

Selon la théorie VSEPR (Valence Shell Electron Pair Repulsion), les paires électroniques (liantes et non liantes) s'éloignent pour minimiser les répulsions. Dans le cas de l'ammoniac, il y a 
quatre régions électroniques autour de l'azote~: trois liaisons \ce{N-H} et une paire d'électrons libre. Ces régions adoptent une géométrie tétraédrique. Cependant, comme seule la paire 
d'électrons libre ne contribue pas à la position des atomes, la forme moléculaire résultante est trigonal pyramidale.

\subsection*{5. Déterminer la représentation de Cram de cette molécule}

La représentation de Cram met en évidence la structure tridimensionnelle de l'ammoniac. Elle montre l'atome d'azote central avec les trois atomes d'hydrogène disposés dans un plan et la paire 
d'électrons libre dirigée vers le haut.

\chemfig{
    N(-[:90]H)(-[:-150]H)(<[:-60]H)
}

\subsection*{Conclusion}

L'ammoniac \ce{NH3} est une molécule à géométrie trigonal pyramidale. L'azote central est lié à trois atomes d'hydrogène par des liaisons covalentes simples, et la présence d'une paire 
d'électrons libre influe sur la forme tridimensionnelle de la molécule, expliquée par la théorie VSEPR. Les schémas de Lewis et de Cram illustrent clairement cette structure.


\end{document}
